% To compile single chapters put a % symbol in front of "\comment" and "}%end of comment" below 
%    and take off the % symbol from "\end{document" at the bottom line. Undo before compiling
%    the complete thesis
%%Note: You can only use \section command, you are not allowed, per TTU Graduate School, use
%%\subsection command for ghigher level subheadings. At most level 2 subheadings are allowed.

\chapter{Conclusions and Future Work}
\label{Conclusions and Future Work}

URE can provide valuable device classification and characterization insights for many applications from NILM to condition-based maintenance. URE processing algorithms often require subject matter expertise to tailor transforms and feature extractors for the specific electrical device of interest.  DASP was presented as a method for projecting aligned signal dimensions, such as frequency harmonics, frequency spacings, signal modulations, that are inherent to the physical implementation of the vast majority of commercial electronic devices, thus removing the need for an intimate understanding of the underlying physical circuitry and the URE generation mechanism. In addition, methods for processing DASP images, extracting statistical features from DASP images, and direct learning from DASP images using CNNs were detailed and tested using a data set of URE captures from commercial electronic devices.

The ability to classify an electronic device's URE using DASP generated features was demonstrated with one-versus-all accuracies approaching $100\%$ for the LDA and k-NN learning methods using statistical-based features, as well as using CNNs to learn directly from DASP images.  The LDA and CNN learning methods were adapted to multi-class all-versus-all classification and although the accuracies did not exceed $66.7\%$, precisions greater than $90\%$ were attained by several DASP trained CNNs with the combination of CNN learners reaching  a $97\%$ precision.  The multi-class CNNs were also tested against DASP images derived from clutter device URE and were able to correctly assign clutter to a clutter class at an accuracy of $80\%$ using the combination of DASP CNN learners.  Finally, analysis of the overall testing results showed that the utilization of the unprocessed DASP arrays exceeded their respective scatter, edge, and radon transformed arrays in terms of accuracy and precision, except for the CMASP edge array which exceeded the performance of the unprocessed CMASP array.

Although the utilization of statistical-based features with the LDA and k-NN learning methods performed well, the ability to learn directly from the DASP images with CNNs provided significantly improved results.  With the simplest of CNN architectures, using only $6$ layers, one-versus-all classification accuracies across all DASP-trained CNN learners reached an average accuracy of $99\%$, however the $6$ layer architecture did not translate well to multi-class applications.  A slightly more complicated CNN with $13$ layers demonstrated the ability of using DASP images to separate multiple classes and to properly identify clutter devices.  Further research in to CNN architectures for DASP image processing is warranted, especially as it pertains to multi-class and clutter classification problem sets.  For instance, a CNN architecture that allows for simultaneous training on all DASP images in multiple parallel streams would significantly improve performance, as demonstrated by \cite{Ciregan2012, Li2015}, and alleviate the need for voting schemes across individual learners.  Additionally, the ability to handle multiple labels per image would allow for more tailored and thorough training with overlapping confounders, devices, and clutter to provide better performance in more realistic commercial or residential URE environments; however, research in multi-label CNNs is relatively new \cite{Wei2016, Wu2015, Gong2013} and primarily focuses on social media image annotation.  

The DASP algorithms presented in Chapter \ref{DASP Algorithm Development Chapter} were not all-encompassing and the continued exploration and development of dimensional alignment algorithms is warranted given the performance of the HASP, MASP, CMASP, FASP, and SCAP algorithms.  Dimensional alignments of phase, wavelet, or chirp-based features have not been fully explored and could provide further insights into URE characteristics and provide additional features for class separation in multi-class classification applications.  Finally, optimization of the DASP parameter values, such as frequencies and bandwidths, should be explored for different devices or classes of devices.  DASP parameter optimization could be accomplished with a supervisory learner, such as a genetic algorithm, wrapped around the CNN learner; however, training and testing would require a significant amount of time and compute resources.


