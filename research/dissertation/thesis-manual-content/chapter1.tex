\chapter{The Essentials}
\label{chap:TheEssentials}

\section{Purpose of the Guide}
\label{sec:PurposeOfTheGuide}

This guide is designed to be a basic source of information for
the\-sis/dis\-ser\-ta\-tion preparation. It establishes the technical
parameters within which you should work, such as quality of paper,
number of copies to be submitted, margins, and the sequence of pages
within the manuscript. Since most of you will publish during and after
your graduate education, this guide encourages the use of leading
professional publications to help establish specific formatting
convention. You are encouraged to use publications within your
field---journals and textbooks---to assist you in establishing
bibliographic form, use of number, and other conventions that are
discipline oriented.  However, the application of this theory is not
simple. You must understand the various elements of a manuscript and
general publication formatting requirements in academic publishing.
Although knowledge and use of publication formatting is essential, the
regulations established by this guide always take precedence.

You should use style handbooks such as the most recent editions of the
\textit{MLA Handbook for Writers of Research Papers} (English)
\cite{gibaldi1988}, \textit{Publication Manual of the American
  Psychological Association} (Education) \cite{apa1983}, \textit{CBE
  Style Manual (Biology)} \cite{cbe1983}, \textit{Form and Style}
(Arts \& Sciences, Engineering, Education) \cite{campbell1990},
\textit{The Chicago Manual of Style} \cite{chicago1982}, and
\textit{Harbrace College Handbook} \cite{hodges1990} as resources for
basic style and grammar. In contrast, you should never use previously
accepted theses and dissertations as the final guide to style.
Examples taken from other theses may be out of context or may be
incorrect. The existence of a particular style or usage in a
previously accepted thesis does not establish a precedent for its
continuation.
% The following two nomenclature entries are intended for the thesis manual.
% Best practice is to add \nomenclature commands with first use of a term.
\nomenclature{MLA}{Modern Language Association}
\nomenclature{CBE}{Council of Biological Editors}

By accepting your thesis or dissertation and awarding the degree,
Tennessee Technological University places its academic reputation on
the line. The content of your manuscript is carefully evaluated by
experts in your field. The format requirements presented in this guide
are imposed to ensure an appropriate academic appearance of your
manuscript.


\section{Ethical Standards}
\label{sec:EthicalStandards}

Since conferral of a graduate degree implies professional integrity
and knowledge of scholarly methods, there are three areas in which you
as a graduate student should be particularly cautious:

\begin{itemize}
\item proper acknowledgment of cited works
\item the proper use of copyrighted material
\item the proper reporting of work where research compliance is required
\end{itemize}

\subsection{Plagiarism}
\label{sec:Plagiarism}

\textit{Merriam-Webster's Collegiate Dictionary} \cite{webster1993}
defines plagiarism as ``steal\-[ing] and pass\-[ing] off ideas or
words of another as one's own'' and ``the use of a created production
without crediting the source.'' ``You must acknowledge all material
quoted, paraphrased, or summarized from any published or unpublished
work.  Failing to cite a source, deliberately or accidentally, is
plagiarism'' \cite[424]{webster1993}. If you use the exact words of
your source, they must be enclosed in quotation marks and the source
cited; if you do not use the exact words but paraphrase or summarize
the source, it still must be cited.  When involved in collaborative
research, you should exercise extreme caution to avoid questions of
plagiarism. If in doubt, check with your major professor and the
Graduate School about the project. Plagiarism will be investigated
when suspected and prosecuted if established.


\subsection{Copyright}
\label{sec:Copyright}

If you use copyrighted material in a limited way, it is usually
unnecessary to seek permission to quote. If, however, you use material
from a copyrighted work to the extent that the rights of the copyright
owner might be violated, you must obtain permission of the owner. In
determining the extent of a written work that may be quoted without
permission, you should consider the proportion of the material to be
quoted in relation to the substance of the entire work. According to
\textit{The Chicago Manual of Style} \cite{chicago1982}, ``A few lines
from a sonnet, for instance, form a greater proportion of the work
than do a few lines from a novel. Use of anything in its entirety is
hardly ever acceptable'' (p. 124). In no case should you copy a
standardized test of similar material and include it in a
the\-sis/dis\-ser\-ta\-tion without written permission. According to
Circular 21 (Reproduction of Copyrighted Works by Educators and
Librarians, p. 11) \cite{loc1988}, ``\textellipsis{}the following
shall be prohibited: \textellipsis{} There shall be no copying of or
from works intended to be `consumable' in the course of study or of
teaching.  These include workbooks, exercises, standardized tests and
test booklets and answer sheets and like consumable material.'' The
publisher usually has the authority to grant permission to quote
excerpts from the copyrighted work or can refer requests to the
copyright owner or designated representative.  The copyright owner may
charge for permission to quote. You should credit permissions with the
acknowledgments, and the source should appear in the
Bibliography\footnote{Some fields alternatively use Literature Cited,
  References, or Works Cited.}.


\subsection{Federal and State Regulations}
\label{sec:FederalAndStateRegulations}

Compliance with federal and state regulations governing the use of
human subjects, animal care, radiation, legend drugs, recombinant DNA,
or the handling of hazardous materials/wastes in research is monitored
by a number of regulatory agencies. Because of these regulations,
research compliance is another area of importance to you as a graduate
student and to the conduct of your research. Tennessee Technological
University requires you to verify that you have complied with the
appropriate approval procedure(s) prior to the initiation of the
thesis- or dissertation-related research, if approval is relevant to
the research. If your research involves any of the areas mentioned
above, you should determine what compliance is required by the school
(available in the Office of Research).


\section{Definitions}
\label{sec:Definitions}


\subsection{Typeface or Font}
\label{sec:DefTypefaceOrFont}

These terms apply to all the features available within a ``type''
family. For many printers, typeface includes bold, italic, and the
various sizes of any named type (Helvetica, Times Roman, New York,
Geneva, etc.).


\subsection{Text}
\label{sec:DefText}

In the discussion of formatting, text is used as a generic term to
designate the main body of the the\-sis/dis\-ser\-ta\-tion and to
distinguish this element from preliminary pages, references, tables,
figures, and appendices.


\subsection{Preliminary Pages}
\label{sec:DefPreliminaryPages}

Sometimes called ``front matter,'' preliminary pages serve as a guide
to the contents and nature of the manuscript \cite{chicago1982}. The
approval or acceptance sheets, as part of the preliminary pages,
confirm acceptance by the committee members acting for the department,
and the Dean of Graduate Studies, acting for the university or
college.


\subsection{Table}
\label{sec:DefTable}

A table consists of numbers, words, or both, and presents information
that is separated into columns. Tabular information allows you, the
author, to convey information to a reader in a structured format.


\subsection{Figure}
\label{sec:DefFigure}

Any diagram, drawing, graph, chart, map, photograph, or material that
does not fit into the restricted format for a table is a figure.
Figures generally show relationships or illustrate information rather
than present precise data.


\subsection{Appendix}
\label{sec:DefAppendix}

An appendix is generally a ``catch-all'' for supplementary material to
the the\-sis/dis\-ser\-ta\-tion. In some cases, tables and/or figures
are placed in an appendix to avoid interrupting the text. An example of an appendix can be seen in Appendix~\ref{chap:sample_tables}.

%%% Local Variables: 
%%% mode: latex
%%% TeX-master: "thesis"
%%% End: 
