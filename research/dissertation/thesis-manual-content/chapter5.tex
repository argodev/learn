\chapter{Technical Pointers}
\label{chap:TechnicalPointers}

Computer use has enabled you to assume responsibility for all aspects
of the\-sis/dis\-ser\-ta\-tion preparation, allowing you to function
as author, editor, and publisher of your manuscript. With this freedom
has come the full responsibility of ensuring that the content is
accurate, grammar and mechanics are acceptable, and all elements of
formatting are handled correctly. The purpose of this chapter is to
provide some pointers on technical production and to address some
common production problems.

\section{Appearance}
\label{sec:Appearance}

The element that contributes most to the attractiveness of a
manuscript is consistency. Consistency in formatting means that you
establish and adhere to a series of conventions or protocols regarding
spacing, heading sequencing, and other aspects of appearance to guide
readers through the manuscript visually, thus enabling them to
concentrate on the content. Consistency in the\-sis/dis\-ser\-ta\-tion
production is especially critical, since it determines in part the
committee reaction to content and, ultimately, acceptance of the
manuscript by the Graduate School.

\section{Content}
\label{sec:Content}

\subsection{Taped Copy}
\label{sec:TapedCopy}

You should avoid wasting valuable time attempting to force the
computer to solve a printing problem when quicker and easier solutions
exist. If not everything to be included in a thesis or dissertation is
on your disk, you must use alternative methods to transfer the image
to a ``working copy,'' such as taping the material to the page.
Examples include material from other sources, photographs, tables, or
other material too large for a standard page. Below are guidelines to
help in taping material--an alternative method of dealing with
noncomputerized material:
\begin{itemize}
\item Prepare tape-up sheets for any material that must be
  repositioned or reduced. Tape-up sheets will have the page number,
  title, and source (if needed) printed in proper position in
  preparation for the material to be taped into place. For pages that
  need only the number, you can create tape-up pages as part of the
  body of the manuscript. All software packages have a means of
  terminating a page at a specific point and advancing to a new page.
  Repeating this will create an empty page, numbered in sequence with
  the rest of the manuscript.
\item For reductions, note that the maximum size of the image area,
  including page number, is 6 by 9 inches. Black and white contrast
  must be sharp. Position of the image on the reduced page is
  unimportant, because the image will be cut out and placed on the
  tape-up page.
\item Trim away nonimage area so that the image can be taped into
  place on the tape-up sheet, using transparent (not cellophane) tape.
  Tape fully all four sides of the image to screen out shadow lines.
  This will become the master copy.
\end{itemize}

\subsection{Photographs}
\label{sec:Photographs}

There are at least six methods for including photographs in your
thesis or dissertation. Each methods differs in quality and cost, and
each requires different handling.
\begin{itemize}
\item With the high-quality reproduction capability of the newer
  copiers, some of which have an automatic screening mode for
  photographs, it is often possible to mount an original photograph on
  a tape-up sheet and have it copied onto 25 percent cotton paper
  without any charge other than the normal copying fee.
\item Individual photographic prints can be mounted in each copy using
  permanent photomount spray adhesive. If you select this option,
  prepare the tape-up sheets and one copy of the photographs trimmed
  approximately 1/8 inch smaller than the other prints. Tape the
  trimmed photographs on all four sides onto the tape-up sheet and
  insert the page into the master copy. Each time you copy the master
  copy, the photographs are also copied. Cost depends on the number of
  negatives and copies purchased. Quality depends on the quality of
  the original photograph.
\item Many students with darkroom access use full-page-size 8.5$\times$11
  inch photographic paper with an image area of 6$\times$9 inches
  (standard margins). Double weight glossy paper is
  recommended for preservation and crisp image. If you select this
  option, print the title and other information on a legend page,
  which precedes the actual photograph, and mount an address label on
  the back of the photograph, one inch down and one inch in from the
  right edge (with the photograph facing downward). Type the label as
  shown below. Give page numbers to both the legend page and the
  photographic page; in the List of Figures, the number shown is that
  of the legend page.  There should be no printing on the front of the
  photograph. The cost of this process depends on whether the darkroom
  work is done by you or by a professional agency. The paper may have
  to be ordered in advance (often 11$\times$14 inch sheets are cut
  down to 8.5$\times$11 inches). The detail quality is excellent.
  \begin{center}
    \begin{tabular}[h]{|c|} \hline
      Figure \# \\
      Page \# \\
      Last Name, Year \\ \hline
    \end{tabular}
  \end{center}
\item Halftone prints are made of each photograph and mounted onto
  paste-up pages. The PMT (photo-mechanical transfer) process screens
  the halftone image and converts it into dots, which can then be
  copied. Generally a dot density of 85 lines per inch gives the best
  image on most copiers. The quality of reproduction is comparable to
  that of a newspaper and probably would not be satisfactory for
  scientific applications. The cost is relatively low, since as many
  photographs as will fit on a sheet of PMT material can be made in
  one shot.
\item Many students use scanners to reproduce photographs, making them
  part of the computer-contained manuscript. Essentially, the scanner
  performs the same function as the PMT process and converts the
  photograph to dots, which are printed as graphics. Fine detail may
  be lost, but the overall image is attractive and copies well.
\item Offset printing is a final option. The process is done by
  full-service print shops and requires the processing of two
  negatives---one for the printed copy and one for the halftone
  photograph. Done well, this process produces excellent quality in a
  form that will last as long as the paper on which it is printed. The
  expense, however, may limit its use in the\-sis/dis\-ser\-ta\-tion
  production.
\end{itemize}

%%% Local Variables: 
%%% mode: latex
%%% TeX-master: "thesis"
%%% End: 
