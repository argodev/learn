\chapter{Thesis/Dissertation Elements and Style}
\label{chap:Thesis/DissertationElementsAndStyle}

\section{Preliminary Pages}
\label{sec:PreliminaryPages}

Figure \ref{fig:ArrangementOfThesisParts} shows the sequence and
numbering scheme of the various the\-sis/dis\-ser\-ta\-tion parts.
Samples of all preliminary pages can be found at the start of this
document.
\begin{figure}[tbp]
  \centering
% Original version of table
%  \begin{tabular}{|p{3.0in}|p{2.0in}|} \hline
%    \textit{Thesis/Dissertation Parts} & \textit{Page Assignment} \\ \hline \hline
%    \textbf{Abstract} & No page number assigned \\ \hline
%    \textbf{Title Page} & Small Roman numeral (Assigned, not typed) \\ \hline
%    Copyright Page & \multirow{9}{*}{Small Roman numeral (Typed)} \\ \cline{1-1}
%    \textbf{Approval Sheet} & \\\cline{1-1}
%    \textbf{Statement of Permission to Use} (Master's theses only) & \\ \cline{1-1}
%    Dedication page & \\ \cline{1-1}
%    Acknowledgments & \\ \cline{1-1}
%    \textbf{Table of Contents} & \\ \cline{1-1}
%    \textbf{List of Tables} (if 2 or more) & \\ \cline{1-1}
%    \textbf{List of Figures} (if 2 of more) & \\ \cline{1-1}
%    \textbf{List of Symbols and/or Abbreviations} (if needed; may be included as an appendix) & \\ \hline
%    \textbf{Body of thesis} (divided into chapters or parts) & \multirow{6}{*}{Arabic numerals, starting with 1} \\ \cline{1-1}
%    \textbf{Separation sheet} & \\ \cline{1-1}
%    \textbf{Bibliography} & \\ \cline{1-1}
%    \textbf{Separation sheet} (if an appendix or appendixes follow) & \\ \cline{1-1}
%    Appendix & \\ \cline{1-1}
%    \textbf{Vita} & \\ \hline
%    \multicolumn{2}{|p{5.5in}|}{Parts in \textbf{bold type} are required; all others are optional.} \\ \hline
%  \end{tabular}
% Booktabs version of table
  \begin{tabular}{p{3.0in}p{2.0in}}
    \textit{Thesis/Dissertation Parts} & \textit{Page Assignment} \\ \toprule
    \textbf{Abstract} & No page number assigned \\ \midrule
    \textbf{Title Page} & Small Roman numeral (Assigned, not typed) \\ \midrule
    Copyright Page & Small Roman numeral (Typed) \\
    \textbf{Approval Sheet} & \\
    \textbf{Statement of Permission to Use} (Master's theses only) & \\
    Dedication page & \\
    Acknowledgments & \\
    \textbf{Table of Contents} & \\
    \textbf{List of Tables} (if 2 or more) & \\
    \textbf{List of Figures} (if 2 of more) & \\
    \textbf{List of Symbols and/or Abbreviations} (if needed; may be included as an appendix) & \\ \midrule
    \textbf{Body of thesis} (divided into chapters or parts) & Arabic numerals, starting with 1 \\
    \textbf{Separation sheet} & \\
    \textbf{Bibliography} & \\
    \textbf{Separation sheet} (if an appendix or appendixes follow) & \\
    Appendix & \\
    \textbf{Vita} & \\ \bottomrule
    \multicolumn{2}{p{5.0in}}{Parts in \textbf{bold type} are required; all others are optional.} \\
  \end{tabular}
  \caption{Arrangement of Thesis/Dissertation Parts}
  \label{fig:ArrangementOfThesisParts}
\end{figure}
\subsection{Abstract}
\label{sec:Abstract}

You must include an abstract with each copy of the
the\-sis/dis\-ser\-ta\-tion submitted to the Graduate School. Although
the content of the abstract is determined by you and your graduate
committee, the following information is appropriate:

\begin{enumerate}
\item a short statement concerning the area of investigation
\item a brief discussion of the methods and procedures used in
  gathering the data
\item a condensed summary of the findings
\item conclusions reached in the study
\end{enumerate}

There is no word limit on the abstract appearing in the thesis or
dissertation but it must be confined to one page in the typestyle
consistent with the text. All doctoral candidates must provide the
Graduate School with an additional abstract that is limited to 350
words (approximately 35 lines) to be sent to Dissertation Abstracts
International.

\subsection{Title Page}
\label{sec:TitlePage}

You will assign the title page the Roman numeral ``i,'' although the
number does not appear on the page. The date which appears shall be
the month and year of commencement. Your name must appear as you are
registered at the University. The wording and format of the title page
must be exactly as shown in Appendix A.

\subsection{Copyright Page}
\label{sec:CopyrightPage}

You will include a copyright page only if the manuscript is being
formally copyrighted (Appendix A). You will find additional
information about copyrighting in Chapter
\ref{chap:BringingItToFruition}.

\subsection{Approval Sheet}
\label{sec:ApprovalSheet}

Each of the copies of the thesis/dissertation submitted to the
University must have an approval sheet using the exact wording and
format shown in the front matter of this manual. This sheet must be on
the same brand and weight of cotton paper and be in the same base
typeface as the remainder of the thesis/dissertation. The name used on
the approval sheets and title page must be that under which you are
registered at the University. Although the approval sheets may be
copies, the committee signatures must be originals. Black ink is
recommended for the original signatures. The number of signature lines
must equal the number of committee members. The major and degree to be
awarded must be exactly those to which you were admitted officially by
the Graduate School. Majors and degrees can be found in the
University's graduate catalog. Number the approval sheet.

\subsection{Statement of Permission to Use}
\label{sec:StatementOfPermissionToUse}

For Master's theses, the Statement of Permission to Use allows the
University Library to provide copies of a thesis for academic use
without securing further permission from you. Unlike dissertations,
theses are not microfilmed, so access to them is limited to that which
can be provided by the Library. You must include with each of the
copies of the thesis submitted to the University a Statement of
Permission to Use on the same brand and weight of paper and in the
same base typestyle. This statement is in addition to optional
copyrighting of the thesis. It follows the approval sheet and is
assigned a page number.

\subsection{Dedication Page}
\label{sec:DedicationPage}

If you wish to dedicate the manuscript, include the dedication
statement at this point.

\subsection{Acknowledgments}
\label{sec:Acknowledgments}

You should use the acknowledgments to thank those who have helped in
the process of obtaining the graduate degree. Also, list permissions
to quote copyrighted material here, as well as acknowledgments for
grants and special funding.

\subsection{Table of Contents}
\label{sec:TableOfContents}

The Table of Contents may vary in style and amount of
information included. However, you must include List of Figures, List
of Tables, List of Symbols, chapter or part titles, the Bibliography,
the Appendix(es), if any, and the Vita. The page numbers for the
Bibliography and the Appendix(es) are the numbers assigned to the
separation sheet preceding each of these items. All headings and
subheadings must be listed in the Table of Contents.

\subsection{List of Tables/List of Figures}
\label{sec:ListOfTables/ListOfFigures}

If there are two or more tables, you must include a List of Tables.
Similarly if there are two or more figures, you must include a List of
Figures. There must be separate lists for tables and figures. Include
in the appropriate list any tables or figures appearing in the
appendix(es). Be sure that each title is different from the other
titles, and that the wording of all titles entered in the lists is
exactly as it appears on the table or figure.  This includes the
information up to the first terminal punctuation.  You do not include
additional explanatory information in the list.

\subsection{List of Symbols/List of Abbreviations/Nomenclature}
\label{sec:ListOfSymbols/ListOfAbbreviations/Nomenclature}

You should make the title of this section reflect its content. You may
use this section to define specialized terms or symbols, or you may
place such information in an appendix.

\section{Text}
\label{sec:Text}

\subsection{Divisions}
\label{sec:Divisions}

This manual has been written in the format described here\-in. You
must divide the man\-u\-script into a logical scheme that you follow
consistently throughout the work. Chapters are the most common major
division; parts are also permissible. Examples of chapter and part
headings are shown in Appendix B. For a discussion of divisions into
``parts,'' see Chapter \ref{chap:SpecialProblemsAndConsiderations}.

Number each chapter or part consecutively and begin on a new page. A
division entitled \textbf{INTRODUCTION} may be the first numbered
chapter or part. Chapter or part titles are primary divisions of the
entire manuscript and are not part of the subdivision scheme.

\subsection{Subdivisions}
\label{sec:Subdivisions}

You may use either the format and order of subdivisions that are
described in this manual or the numerical decimal system of
identifying heading and subheading. The subdivisions within a chapter
or part do not begin on a new page unless the preceding page is
filled. First and second level subdivisions are always preceded by an
extra blank line to indicate to the reader a major shift in subject.
\textbf{Never} have \textbf{only one} subdivision at any level.

\subsubsection{Centered head}
\label{sec:CenteredHead}

If there is not room for the complete heading and at least two lines
of text at the bottom of a page, begin the new subdivision on the next
page. If a chapter contains only one level of subdivision, use the
centered head. Type the first letter of each word in caps, place it in
bold type (or underline if bold is not available), and center it four
inches from the right edge of the page. Place it two blank lines (line
spacing = 3) below the preceding text and two blank lines above the
text which follows. Double-space (line spacing = 2) in an inverted
pyramid format a centered head that is longer than four inches.
\textbf{If a second level of subdivision immediately follows the
  centered head, use only one blank line (line spacing = 2) between
  the two subheadings.}

\subsubsection{Freestanding sidehead}
\label{sec:FreestandingSideHead}

If a chapter makes use of two levels of subdivision, then a
freestanding sidehead is the second subdivision. Position the
freestanding sidehead flush with the left margin (see Margin Settings
and Justification), two blank lines below the preceding text
(\textbf{double space if preceded by a centered head}) and two blank
lines above the text that follows. Capitalize the first letter of each
major word.  Place the sidehead in bold type; there is no end
punctuation. If the heading is longer than 2.5 inches, use a second
line. Indent the second line two spaces and double space between the
two lines.

\subsubsection{Paragraph sidehead}
\label{sec:ParagraphSidehead}

A third subdivision is indicated by a paragraph sidehead which is
subordinate to both the centered head and the freestanding sidehead.
Place the paragraph sidehead a single blank line below the preceding
text. Indent it like a regular paragraph. Capitalize only the first
letter of the first word. Place the heading in bold type, followed by
a period, and in every instance begin the text on the same line.

\subsection{Quotations}
\label{sec:Quotations}

You must give full credit for every quotation or paraphrase used. A
carefully worded paraphrase is usually preferable to a long quotation.
Paraphrases are not enclosed in quotation marks. If you use a footnote
to acknowledge a source, its' superscript normally follows the final
punctuation of the material cited; however, you should place the
superscript at the end of a sentence if only the sentence is
referenced.

Quotations are used when it is desirable to reproduce literary
material in exact detail. Quotations which are not over three lines
long are usually enclosed in quotation marks and are place within the
text. When quotations are longer, they are usually set off from the
test in a separate paragraph or paragraphs and single-spaced. Follow
the guidelines of conventional practice in your discipline.

\subsection{References Within Text}
\label{sec:ReferencesWithinText}

Notes documenting the text and corresponding to a superscripted number
in the text are called footnotes when they are printed at the bottom
of the page \cite{chicago1982}. This format is only used occasionally
and has generally been replaced by references. References usually
consist of information in parenthesis or square brackets within the
text. Two common methods of referencing are (1) to use author's name
and date of publication, as in (Smith, 1990), or (2) to assign numbers
to the bibliographical entries and insert the corresponding number for
the authors as they are cited in the text, as in Smith [95]. The
purpose of references is to guide the reader to the corresponding
entry in the Bibliography, where complete information is available.
Footnotes or reference notes collected at the end of each chapter or
part (end note) are not acceptable. In microfilm or other electronic
format, large numbers or pages are reproduced on a single sheet of
film, making end notes difficult for the reader to locate. You must
determine the form, style, and contents of footnotes or reference
notes by what is generally accepted in your field of study.

Most of the popular word processing applications have a footnote
feature that provides automatic formatting and placement of footnotes
at the bottom of the page. For disciplines using that convention, the
formatting provided by the software application would be acceptable.

\section{Tables and Figures}
\label{sec:TablesAndFigures}

\subsection{General Information}
\label{sec:GeneralInformation}

\subsubsection{Titles}
\label{sec:Titles}

Since tables and figures are separate entities, you must number them
independently. Each table or figure must have a unique title
descriptive of its contents. This title appears at the top of the
table and at the bottom of the figure. Give figures containing parts a
general title, after which you may break the figure down into ``A''
and ``B'' parts. For multiple-part figures, you may integrate the
title, with titles for each part as part of the general figure title,
or composite, with no reference to the individual parts. No two
figures may have exactly the same title. The formatting of the titles
must be consistent for all tables and figures.

\subsubsection{Numbering}
\label{sec:Numbering}

You may number tables and figures in one of several ways. Three of the
most common numbering schemes are:
\begin{itemize}
\item to number consecutively throughout the manuscript, including the
  appendix(es), using either Roman or Arabic numerals
\item to number consecutively within chapters, parts, or appendixes,
  with a prefix designating the chapter/part/appendix (e.g., 3-1, 3-2
  . . . 4-1, 4-2, A-1, B-1)
\item to establish a consecutive numbering system for the body of the
  manuscript and a different one for the appendix(es) (e.g., 1, 2, 3
  for text and A-1, A-2, A-3 for appendix)
\end{itemize}
The style of numbering must be consistent.

\subsubsection{Placement within the body of the manuscript}
\label{sec:PlacementWithinTheBodyOfTheManuscript}

You must make each table or figure immediately follow the page on
which it is first mentioned (except as noted in the next paragraph),
and you must refer to all tables and figures by number, not by
expressions such as ``the following table/figure.'' When more than one
table or figure is introduced on a page of text, each follows in the
order mentioned. You may find it convenient to assign tables and
figures pages separate from the text to avoid problems in shifting
during last-minute revisions. In degree of importance, tables and
figures are secondary to the text so that the text dictates where the
tables or figures are placed. You must fill all pages with text and in
no case should a page be left significantly short because of the
mention of a table or figure.

You may incorporate within the text a table or figure less than
one-half page in length (approximately four inches), provided it meets
the following conditions:
\begin{itemize}
\item Is in numerical order
\item Is separated from the text by extra space (approximately one-half inch)
\item Is not continued onto a following page
\item Follows its specific mention in the text
\end{itemize}

If tables and figures are integrated with text, you must place them so
that they appear either at the top or the bottom of a page. A mention
on the upper half of a page of text would mean that the bottom half of
the page would be reserved for the table or figure, and a mention in
the bottom half of the page would place the table or figure at the top
of the next page. Always have a balance of no less than one-half page
of text and no more than one-half page of table or figure. If multiple
tables or figures are mentioned together on a page, you may place them
on pages together, provided there is approximately one-half inch
between each. You need not designate as figures small diagrams within
the text, nor designate as formal tables compilations which are no
more than a few lines in length.

\subsubsection{Placement of tables and figures in the appendix}
\label{sec:PlacementOfTablesAndFiguresInTheAppendix}

When all tables and/or figures are in an appendix, you will so state
in a footnote in the body of the text attached to the first mention of
a table or figure; do not repeat this information thereafter. When
only some of the tables and figures are in an appendix, clearly
indicate their location when the items are mentioned in the text
(e.g., Table 1, Appendix A), unless the numbering scheme makes the
location obvious (e.g., Table A-1).

\subsubsection{Horizontal tables and figures}
\label{sec:HorizontalTablesAndFigures}

To accommodate large tables or figures you must sometimes place them
in horizontal (landscape) orientation on the page. The margin at the
binding edge must still be 1.5 inches, and all other margins at least
one inch. The margin at the top of the page and the placement of the
page number must be consistent with the rest of the thesis. Place the
table or figure and its caption so that they can be read when the
thesis is turned 90 degrees clockwise.

\subsubsection{Foldout Pages}
\label{sec:FoldoutPages}

If possible, reduce large tables and figures to fit an 8.5$\times$11 inch
page. If not, you may include in the thesis material on
approved paper larger than 8.5$\times$11 inches, provided the page
itself is 11 inches vertically and is folded properly. The fold on the
right side must be at least one-half inch from the edge of the paper.
The second fold, on the left side, if needed, must be at least 1.5
inches from the binding edge of the paper. The finished page, folded,
should measure 8.5$\times$11 inches. If the page is to be bound into
the thesis or dissertation, the paper submitted to the Graduate School
must be the same brand of 25 percent cotton bond\footnote{See page
  \pageref{sec:PaperAndDuplication} for specific paper requirements.}
as the rest of the manuscript.

\subsubsection{Material in pockets}
\label{sec:MaterialInPockets}

If it is necessary to include a large map, drawing, floppy disk,
videotape, or any other material which cannot be bound, you must
itemize these materials in the Table of Contents and designate them as
being ``In Pocket.'' Affix to the pocket material a label including
number, title, your name, and year of graduation. A pocket for the
material will be attached to the inside back cover of the bound
copies.

It is also permissible to include less bulky material such as a survey
instrument or pamplets in a pocket attached to a sheet of approved
paper with permanent cement. You must treat this material as a figure,
mention it in the text, and give it a number and caption. Observe
caution in using pockets since the material in them is easily lost.

\subsection{Tables}
\label{sec:Tables}

\subsubsection{Typeface}
\label{sec:TableTypeface}

For the table captions you must use the base typeface and size used
for the manuscript. The size of the type within the table may differ,
depending on the ``fit'' of the information within the margins.

\subsubsection{Required components}
\label{sec:TableRequiredComponents}

Since tables consist of tabulated material or columns, the use of
ruling or horizontal lines in tables helps the reader distinguish the
various parts of the table. Vertical lines are accepted but not
required. One of the characteristics that identifies tabulated
material as a table is the presence of at least the following three
horizontal lines:
\begin{itemize}
\item The table opening line, which appears after the table caption
  and before the columnar headings
\item The columnar heading closing line, which closes off the headings
  from the main body of the table
\item The table closing line, signaling that the data are complete
\end{itemize}
Anything appearing below the closing line is footnote material.

Tables must have a least two columns which carry headings at the top
as brief indications of the material in the columns
\cite[329]{chicago1982}. The headings appearing between the table
opening line and the column heading closing line must apply to the
entire column down to the table closing line. It is never appropriate
to change columnar headings on continued pages. One method of avoiding
a problem is to use subcolumnar heads, which are headings that appear
below the column heading closing line, cut across the columns of the
table and apply to all the tabular matter lying below it
\cite[330]{chicago1982}.

\subsubsection{Continued tables}
\label{sec:ContinuedTables}

You may continue tables on as many pages as necessary, provided the
columnar headings within the columnar block remain the same. Repeat
the columnar block for each page. Do not repeat the table caption, but
indicate continuation pages with the designation: Table
$\underline{\quad}$ (Continued). You may reduce tables too large to
fit within margins.  See Chapter \ref{chap:BringingItToFruition} for
hints on technical production.

\subsubsection{Table footnotes}
\label{sec:TableFootnotes}

Footnotes to tables consist of four different categories
\cite{turabian1987}:
\begin{itemize}
\item \textbf{Source notes.} If you take the table or data within the
  table from another source,use the word \textbf{Source(s):}, followed
  by the full reference citation, regardless of the format of
  referencing used in the main body of the text. This ensures that if
  that specific page is copied in the future by an interested reader,
  all bibliographic information is contained within the page. Include
  all references in the Bibliography.
\item \textbf{General Notes.} Introduce general notes, which may
  include remarks that refer to the table as a whole, as
  \textbf{Note(s):}.
\item \textbf{Superscript notes.} For notes to specific parts of the
  table use superscripts (letters for tables consisting of numbers;
  numerals for tables consisting of words; symbols if letters or
  numbers might be mistaken for exponents) that are attached to the
  part of the table to which they apply.
\item \textbf{Level of probability notes.} For a table containing
  values for which levels of probability are given, use asterisks. Use
  a single asterisk for the lowest level of probability, two for the
  next higher, etc. \cite{chicago1982}.
\end{itemize}

\subsection{Figures}
\label{sec:Figures}

\subsubsection{Typeface}
\label{sec:FigureTypeface}

Since figures are considered illustrations, regardless of the nature
of their content, any print that is part of the figure can be in any
neat and legible typeface. You must use the same base typeface and
size for the figure caption and page number as in the rest of the
manuscript because this material is considered to be part of the
typeset body of the manuscript (see Chapter
\ref{chap:BringingItToFruition}).

\subsubsection{Legends}
\label{sec:Legends}

You may place explanatory material for figures within the figure,
either above or below the caption, or continue it after the period
following the caption. If a figure has a long caption and/or legend
which must be placed on a separate sheet because of the size of the
figure, place this page immediately before the figure. The page number
assigned to the caption page is considered to be the first page of the
figure.

\subsubsection{Continued figures}
\label{sec:ContinuedFigures}

You may continue onto other pages a figure containing several related
parts too large to be included on a single page. The first page
contains the figure number and complete caption, and subsequent pages
contain the remainder of the figure and the designation: Figure
$\underline{\quad}$ (Continued).

\subsubsection{Figure footnotes}
\label{sec:FigureFootnotes}

Footnotes to figures consist of two different categories
\cite{turabian1987}:
\begin{itemize}
\item \textbf{Source notes.} If the figure or information within the
  figure is taken from another source, use the word
  \textbf{Source(s):}, followed by the full reference citation,
  regardless of the format for referencing used in the main body of
  the text. This ensures that if that specific page is copied in the
  future by an interested reader,all bibliographic information is
  contained within the page.  If you have made changes in a figure
  from another source, so indicate by using the phrase ``Adapted from
  \textellipsis{}.''
\item \textbf{General notes.} Introduce general notes, which may
  include remarks that refer to the figure as a whole, as
  \textbf{Note(s):}.
\end{itemize}

You must include all references in the Bibliography. 

\subsection{Equations}
\label{sec:Equations}

The most recent edition of \textit{The Chicago Manual of Style}
\cite{chicago1982} is a good resource. Generally, it is expected that
all equations will be typewritten or printed in the final copy. With
some word processing programs (e.g., Word, WordPerfect) you can create
equations that contain any number of special characters and symbols.
When questions arise concerning the placement of equations, proper
spacing, and indentations, feel free to consult with the
the\-sis/dis\-ser\-ta\-tion consultant in the Graduate School. The
following general rules apply in the use of equations:
\begin{itemize}
\item Align on operational signs equations that have more than one
  line.
\item Center equations between the left- and right-hand margins.
\item Do not break at the end of a line a short equation in the text;
  rather you should ``space out'' the line so that the equation will
  begin on the next line; or you may center the equation on a line by
  itself.
\item Set connecting words of explanation such as \textit{hence},
  \textit{therefore}, and \textit{similarly} at the left-hand margin
  either on the same line with the equation or on a separate line (if
  used with a numbered equations). Do not use commas following these
  words.
\item Number displayed equations (those set on separate lines)
  consecutively throughout each chapter, flush with the right margin.
\item Follow equations that end a sentence with a period, normally on
  the line of type which conludes the equation. For equations that
  have several horizontal lines, align the period with the equal sign.
  The use of the period should be regarded as an aid to clarity.
\end{itemize}

\subsection{Bibliography}
\label{sec:Bibliography}

%%% XXX %%%

You must include a list of materials used in the preparation of the
manuscript of the thesis/dissertation. This may consist only of
references cited in the text or it may include works consulted as
well. The list is preceded by a numbered page with the title centered
vertically and horizontally (see Appendices H--K). The purpose of
listing the citations is threefold:
\begin{enumerate}
\item to serve as an acknowledgement of sources
\item to give readers sufficient information to locate the volume
\item in the case of personal interviews or correspondence, to save
  readers the trouble of attempting to locate material that is not
  available
\end{enumerate}

If your appendix contains references, the appendix \textbf{must}
precede the bibliography. Follow the format for the citations used in
your field of study.

\subsection{Appendix}
\label{sec:Appendix}

An appendix (appendixes or appendices), if included, is preceded by a
numbered page with the designation centered vertically and
horizontally between the margins. Place original data and
supplementary materials in the appendix. In some cases, all tables and
figures are included in the appendix(es).

\subsection{Vita}
\label{sec:Vita}

Write the vita, which contains appropriate personal, academic, and
professional information about you, in narrative form. Since copies of
the manuscript will be available to the public, do not include private
information. The vita is the last item in the manuscript and appears
with no preceding separation page.

%%% Local Variables: 
%%% mode: latex
%%% TeX-master: "thesis"
%%% End: 
