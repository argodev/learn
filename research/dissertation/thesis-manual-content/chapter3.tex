\chapter{Formatting}
\label{chap:Formatting}

\section{Typeface and Quality}
\label{sec:TypefaceAndQuality}


\subsection{Typeface or Font}
\label{sec:TypefaceOrFont}

The typeface or font you use affects the physical appearance of your
manuscript more than any other single element. Because of computers
and the availability of laser printers and high-quality dot matrix
printers, typewriters no longer represent the standard by which the
physical appearance of the manuscript is defined. Although typewritten
text is still acceptable, word processing is considered to be the
latest technology.

If a typewriter or standard printer is used, use the basic typeface
(e.g., Letter Gothic, Prestige, Courier) consistently throughout the
thesis. The pitch used must be the pitch for which the type was
designed (i.e., Courier 10 must be set on 10 spaces to the inch and
Prestige Elite must be set on 12 spaces). You may insert symbols not
available on the typewriter by using dry transfer lettering or a
template, or by printing symbols onto drafting film applique from a
printer.

Laser printers provide the opportunity to use different type sizes and
special effects such as bold and italics. Although most laser printers
also have some typewriter styles available as options, the sizes of
the type on a laser printer are often measured in points rather than
in characters per inch. Text is normally most readable in 12-point, so
this size is highly recommended. You may use other sizes for emphasis.

You must consistently follow your styles or conventions used for
special effects throughout the manuscript. If you decide to set
single-spaced quotes in italics or in a smaller type than that used
for the regular text, you must follow that convention for all
single-spaced quotes. Other illustrations of special effects may be
found in journals or textbooks.

The typeface or font selected for text will be the base style or the
``starting point'' for all type selection \cite{wordperfect1988} and
will establish the framework for the entire manuscript. All the
following items must be in the family of type selected as the ``base''
style:
\begin{itemize}
\item Preliminary pages
\item Text
\item Table captions
\item Figure captions
\item Page numbers
\end{itemize}

\subsection{Type Quality}
\label{sec:TypeQuality}

Acceptable type quality for the final \textbf{master} copy is
determined by the following factors:
\begin{itemize}
\item The visual smoothness of the letters
\item Standard uppercase and lowercase letters
\item The presence of descenders (parts of letters that normally
  extend below the line, such as p, q, y)
\item A high-contrast, solid image
\end{itemize}

The printers most commonly used to produce the final master copy are
laser, 24-pin dot matrix, ink-jet, and daisy-wheel printers. You
should confirm the acceptability of other printers with the
the\-sis/dis\-ser\-ta\-tion consultant in the Graduate School. Some
general guidelines for producing acceptable-quality master copy are:
\begin{itemize}
\item install new ribbon, toner cartridge, or ink cartridge
\item clean the printer head or daisy wheel
\item use plain white paper (not 25 percent cotton)
\end{itemize}

\section{Spacing}
\label{sec:Spacing}

Spacing has both aesthetic and utilitarian effects on the appearance
of a page. Vertical spacing determines the number of typed lines that
will fit on a page and can make a manuscript appear either cluttered
or uncluttered, depending on space left between lines. Horizontal
spacing ``tightens up the spaces between certain pairs of letters,
such as WA'' \cite[604]{alfieri1988}, and makes the spacing of
proportional fonts pleasing to the eye.

Most technical decisions about both vertical and horizontal spacing
are determined by the software package. When you select a typeface and
size, the default values for spacing are automatically set. Most word
processing packages then allow you to set the line spacing, using the
predetermined line height as a basis. Single spacing leaves a small
space between two lines of type and double spacing leaves the
equivalent of the height of a line between the two lines of type.

You must double space the general text. You may use single spacing to
set off quoted material and for references and tables. In the event
that an extra blank line is needed (e.g., between chapter number and
title), you should add an additional ``enter,'' doubling the white
space. See Subdivisions, for specific spacing instructions for
headings.

\subsection{Indentations}
\label{sec:Indentations}

Make paragraph indentations uniform throughout the
the\-sis/dis\-ser\-ta\-tion. Indent the paragraph from five to 10 spaces.

\subsection{Widow/Orphan Lines}
\label{sec:WidowOrphanLines}

Avoid single lines of a paragraph at the top and bottom of a page
(widow and orphan lines). If you must divide a paragraph at the bottom
of a page, make at least two lines appear at the bottom and carry at
least two lines to the top of the next page. If there is not room for
a complete heading and at least two lines of text at the bottom of a
page, begin the new subdivision on the next page.

\section{Other Formatting Considerations}
\label{sec:OtherFormattingConsiderations}

\subsection{Margin Settings and Justification}
\label{sec:MarginSettingsAndJustification}

The left margin \textbf{must} be no less than 1.5 inches; the right,
top, and bottom margins no less than 1 inch. All images, including the
page number, must fit within these margins. These margins define the
minimum white space to be maintained on all sides.

A fully justified line of type, regardless of the number of words in
it, is exactly the same length as all other lines \cite{chicago1982}.
This feature is an option in most word processing packages. Either
fully justified or left-justified margins are acceptable. The use of
justified margins must be consistent throughout the manuscript.

\subsection{Pagination}
\label{sec:Pagination}

The Abstract is not assigned a page number. Use small Roman numerals
to number all other pages preceding the text. Although the preliminary
paging begins with the title page, no number appears on that page;
therefore, the following page is page ii. Beginning with the first
page of text, number all pages consecutively throughout the
manuscript, including the Bibliography, Appendix(es), and Vita, with
Arabic numerals. Pagination using letter suffixes (i.e., 10a and 10b)
is not allowed. Number the initials page of any major subdivision
(e.g., the first page of a chapter, division pages) at the bottom,
leaving a margin below the page number of 1 inch from the bottom edge
and centered on 4 inches from the right edge of the page.
\textbf{Update: if you are having difficulty with the placement of the
  page numbers being in two different locations, you may choose to
  place the page number at the bottom center on all pages.} Place the
numbers of other pages in the upper right-hand corner, leaving a
margin of one inch from the top edge and one inch from the right edge
of the page, with the text beginning a double space below. Make sure
that numbers appear on separation sheets.

\subsection{Paper and Duplication}
\label{sec:PaperAndDuplication}

Print or type the master copy on plain white paper. Reproduce the two
copies of the the\-sis/dis\-ser\-ta\-tion submitted to the Graduate
School on 25 percent cotton content, 20 pound weight, white paper. Use
the same brand of paper throughout both copies and for the approval
pages.

%%% Local Variables: 
%%% mode: latex
%%% TeX-master: "thesis"
%%% End: 
