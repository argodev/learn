\chapter{Special Problems and Considerations}
\label{chap:SpecialProblemsAndConsiderations}

The guidelines given in the previous chapters are sufficient for most
the\-ses/dis\-ser\-ta\-tions; however, there are several circumstances
that require additional guidance. This chapter addresses a few of the
more specific questions that may exist in the preparation of your
the\-sis/dis\-ser\-ta\-tion, such as the use of papers that have been
or will be submitted to journals, and the division of unusually long
manuscripts.

\section{Theses/Dissertations in the Form of Journal Articles}
\label{sec:ThesesDissertationsInTheFormOfJournalArticles}

A thesis or dissertation may include articles submitted or about to be
submitted to professional journals. However, some guidelines apply.
You must integrate the individual papers into a unified presentation.
This might be done through an introductory chapter containing, among
other things, a detailed literature review of the type not presented
in journal articles. Additionally, you might use one or more
connecting chapters to expand upon the methodology or the theoretical
implications of the findings presented in the individual articles. You
must adopt a uniform style of headings, reference citations, and
bibliographical format---in compliance with this guide---for the
the\-sis/dis\-ser\-ta\-tion, even though you may have prepared the
individual papers for submission to different journals. You may list
each paper as an individual chapter within the
the\-sis/dis\-ser\-ta\-tion, or you may treat each paper as a part and
follow the multipart format discussed in the next section. If you use
chapter divisions, you will include only one Bibliography (including
all references from the various articles) at the end of the text.
Finally, you may add appendixes to present information not included in
the chapters. Number pages consecutively throughout the manuscript.

\section{Multipart Theses and Dissertations}
\label{sec:MultipartThesesAndDissertations}

With approval of the committee members, you may divide the
the\-sis/dis\-ser\-ta\-tion into parts, rather than sections or
chapters. The use of parts is an effective method of organization when
you have performed research in two or more areas not practical to be
combined into a single presentation or when you wish to maintain
consistent format for journal articles. You may treat each part as a
separate unit, with its own chapters, figures, tables, Bibliography,
and Appendix(es) (if needed). You may combine the Bibliography and
Appendix(ex) at the end, as in the case of
the\-ses/dis\-ser\-ta\-tions in the form of journal articles (see
previous section). In all cases, you must include an abstract or
foreword which provides an overview and summary of the project, and a
single Table of Contents, List of Tables, and List of Figures. Use
consecutive pagination throughout the manuscript, including numbering
of the required separation sheets listing the part number and title
placed before each part.

\section{Two-Volume Theses/Dissertations}
\label{sec:TwoVolumeThesesDissertations}

If a manuscript is more than 2.5 inches in thickness (approximately
500 sheets of 20 pound 25 percent cotton paper), you must divide it as
equally as possible into two volumes not exceeding 2.5 inches each.
You must make the divisions between chapters or major divisions, such
as Bibliography or Appendixes. List the contents for the entire
manuscript in the Table of Contents at the beginning of Volume 1.
Pagination is continuous throughout both volumes. Just prior to
Chapter 1, insert a sheet with VOLUME 1 centered both horizontally and
vertically between margins. Volume 2 opens with a title page followed
by a sheet showing VOLUME 2. Do not assign a page number to either
volume separation sheet.

%%% Local Variables: 
%%% mode: latex
%%% TeX-master: "thesis"
%%% End: 
