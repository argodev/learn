\setcounter{page}{1}    %set correct page here
%%Note: You can only use \section command, you are not allowed, per TTU Graduate School, use
%%\subsection command for higher level subheadings. At most level 2 subheadings are allowed.

\chapter{Introduction}

\bigskip
\setlength{\epigraphwidth}{.6\textwidth}
\begin{epigraphs}
	\qitem{All models are wrong, but some are useful.}%
	{---\textsc{George Box}}
	\centering
\end{epigraphs}
\bigskip

All electronic devices have non-ideal filters, manufacturing design variances, and a variety of other imperfections that can lead to unintended radiated emissions (URE) from clocked signals, frequency mixing, and signal modulations \cite{Stagner2014, Boroyevich2014, Meynard2011, Prvulovic2017}.  The processing of URE can provide a significant amount of information about the equipment that generated the signals, including equipment type \cite{Wang2012} or its current operating condition \cite{Vuagnoux2009}.  URE characterization research can be broadly separated into the areas of 1) Electromagnetic interference (EMI), 2) Non-intrusive load monitoring (NILM), and 3) Information security (IS). EMI research generally focuses on minimizing URE, NILM on understanding URE, and IS on exploiting URE.  URE appears both in the radiated electromagnetic (EM) spectrum and conducted onto the power infrastructure, with the latter typically being where NILM research is applied to load disaggregation for energy efficiency applications and fault detection for condition-based maintenance applications \cite{Harrold1979, Maughan2010, Timperley2017}.   In addition, information security research focuses on understanding emanations from information processing systems and preventing unauthorized access to private or protected information. 

The majority of NILM research does not actually focus on the unintended emissions from equipment themselves, but rather the load profile and voltage perturbations induced onto the power infrastructure with the operation of electrical equipment.  NILM signal processing can be roughly divided into transient and steady state analysis, with both often requiring more than one measurement point such as current and voltage \cite{Dinesh2016}.  Transient analysis works well with large inductive and machine loads due to the current draw and perturbations generated with large equipment start-up; however, these techniques can be confounded by simultaneous turn-on events \cite{Chang2012}.  Steady-state approaches utilize real and reactive power draw of equipment as well as power line harmonic analysis to detect and classify equipment, but often utilize changes in steady-state operations for triggering which is not applicable to always-on devices and can be confounded with a large number of devices and transients \cite{Laughman2003}.  Fault detection for condition-based maintenance applications, such as \cite{Benbouzid2000}, does utilize unintended emanations and, in addition, \cite{Stagner2014} demonstrates that analysis of URE transients associated with switching frequencies could be used as a potential feature for NILM device characterization.  There is a significant amount of literature on switching frequency emanations, but it generally focuses on development of EMI mitigation strategies, such as clock spreading as shown by \cite{Hardin1994} and \cite{Hsiang-Hui2003}; however \cite{Cooke2016} does present a harmonic analysis of USB charger switching frequencies for NILM applications and \cite{Gulati2014} analyzes EMI signatures of switched-mode power supplies for device identification.  Although a significant amount of NILM research has been applied to determining optimal machine learning classifiers, such as neural networks \cite{Srinivasan2006}, decision trees \cite{Gillis2016}, or support vector machines (SVM) \cite{Duarte2012}, the majority of features are derived from the transient and steady state load characteristics. 

IS, or information leakage, research was first presented in the $1960$s \cite{Zaji2014} and URE characterization is a primary focus.  It has been shown that URE can allow an eavesdropper access to keystrokes \cite{Vuagnoux2009}, television images \cite{Kuhn2013, Enev2011}, and even cryptographic information \cite{Hayashi2013a}.  Targeting specific algorithms, software implementations, or hardware implementations may require a deep understanding of the underlying URE generation mechanism or URE characteristics as shown in \cite{Zaji2014}, although new research has shown generic detection and characterization methods for detecting clocked digital circuitry \cite{Stagner2014} and methods for detecting amplitude and frequency modulations within computer emanations \cite{Prvulovic2017}.  A similar intimate understanding of the design and physical layout of hardware circuitry is also required in the EMI research community, as is demonstrated in the plethora of literature dedicated to reducing EMI of voltage converters \cite{Liu2007}, pulse-width-modulation (PWM) generation \cite{Skibinski1999}, and clocking methods \cite{Zhou2011}. 

Whether detecting and characterizing an electrical kitchen appliance for a NILM application or determining the state of a computer algorithm for an information leakage research effort, URE characterization can be viewed as a typical machine learning classification problem.  Though significant progress has been accomplished in terms of load characterization for NILM applications, no approach has been applied to align URE-based features for a significantly reduced feature space.  DASP and associated transformations are presented as a new method for generating features from URE for NILM and associated URE classification applications.  The DASP methodology aligns signal characteristics that are inherent to electrical circuitry to generate features for subsequent machine learning classifiers.  Harmonic, modulation, and frequency spacing signal dimensions were utilized to demonstrate the performance improvements gained by aligning these signal characteristics into a 2D image which is subsequently summed into a 1-D vector for statistical feature extraction or processed directly by a computer vision machine learner. 

A typical commercial or residential facility contains a large number of electrical devices which all generate URE.  A measurement of the electrical infrastructure would therefore contain the superposition of all URE time domain signals, which could significantly complicate and confound a NILM processor.  In addition to characterizing the URE from a single device, DASP generated images were processed by a convolutional neural network (CNN) image recognition machine learner to demonstrate applicability in multi-device and high noise URE environments.  

CNNs are a deep learning framework specifically developed for object recognition and computer vision applications \cite{Krizhevsky2012} and have achieved significant success in the ImageNet Large Scale Visual Recognition Challenge \cite{Simonyan2014, Szegedy2015, He2016}.  CNNs are scale and shift invariant \cite{Scherer2010} allowing for recognition of objects, shapes, or features regardless of location or translation within an image, which makes them well suited for computer vision applications.  The CNN operates on an image by isolating low level features, such as ``door'', ``windows'', and ``chimney'', that, in combination, are identified as a ``house'' at the higher level learning layers \cite{LeCun2015}.  The vast majority of CNN research is applied to the processing of images, mostly optical \cite{Simonyan2014, Szegedy2015, He2016} or medical \cite{Chen2017, Tajbakhsh2016}, with little research currently being applied to synthetic images originating from scientific data, such as DASP; however recent applications in voice spectral image analysis \cite{Qian2016} and particle physics experimental analysis \cite{Aurisano2016, Racah2016} indicate increasing adoption of CNN learning frameworks in new research areas.  The CNN learning process, along with its scalability, flexible architecture, and shift invariance, make it uniquely suited for processing of single device DASP images and DASP images with multiple superimposed URE signatures in particular.     

\section[Dissertation Outline]{Dissertation Outline}

In Chapter \ref{URE Model Development Chapter} a model for the generation and conduction of unintended emanations is derived from the inherent operation of electronic devices.  To evaluate the performance of the DASP algorithms, time domain captures of URE were collected from commercial electronic devices as outlined in Chapter \ref{URE Data Collection Chapter}.  Chapter \ref{DASP Algorithm Development Chapter} describes the development of the Harmonically Aligned Signal Projection (HASP), Modulation Aligned Signal Projection (MASP), Spectral Correlation Aligned Projection (SCAP), Cross-modulation Aligned Signal Projection (CMASP), and Frequency Aligned Signal Projection (FASP) dimensional alignment algorithms.  Chapters \ref{DASP Feature Extraction Chapter} and \ref{Simulation and Testing Configuration} outline the processes for transforming, scaling, and extracting features from DASP generated images and establishes the testing parameters and procedures for the evaluation of the DASP algorithms using the linear discriminant analysis (LDA), k-nearest neighbor (k-NN), and CNN machine learning methods, as described in Chapter \ref{DASP Device Classification Chapter}.
 

%%%%%  Adding nomenclature 
%
%\nomenclature{PSD}{Power Spectral Density}
%\nomenclature{DASP}{Dimensionally Aligned Signal Projection}
%\nomenclature{CMASP}{Cross-Modulation Aligned Signal Projection}
%\nomenclature{HASP}{Harmonically Aligned Signal Projection}
%\nomenclature{HASP-F}{Harmonically Aligned Signal Projection - Fixed Type}
%\nomenclature{HASP-D}{Harmonically Aligned Signal Projection - Decimation Type}
%\nomenclature{FASP}{Frequency Aligned Signal Projection}
%\nomenclature{MASP}{Modulation Aligned Signal Projection}
%\nomenclature{SCAP}{Spectral Correlation Aligned Projection}
%\nomenclature{URE}{Unintended Radiated Emissions}
%\nomenclature{NILM}{Non-intrusive Load Monitoring}
%\nomenclature{LDA}{Linear Discriminant Analysis}
%\nomenclature{k-NN}{K-Nearest Neighbor}
%\nomenclature{CNN}{Convolutional Neural Network}
%\nomenclature{FFT}{Fast Fourier Transform}
%\nomenclature{EM}{Electro-magnetic}
%\nomenclature{IS}{Information Security}
%\nomenclature{EMI}{Electro-magnetic Interference}
%\nomenclature{SVM}{Support Vector Machines}
%\nomenclature{PWM}{Pulse Width Modulation}
%\nomenclature{USRP}{Universal Software Radio Peripheral}
%\nomenclature{TCXO}{Temperature Controlled Crystal Oscillator}
%\nomenclature{GPS}{Global Positioning System}
%\nomenclature{RF}{Radio-frequency}
%\nomenclature{STFT}{Short-time Fourier Transform}
%\nomenclature{LoG}{Laplacian of Gaussian}
%\nomenclature{TIFF}{Tag Image File Format}
%\nomenclature{PCA}{Principal Component Analysis}
%\nomenclature{QDA}{Quadrature Discriminant Analysis}
%\nomenclature{SGDM}{Stochastic Gradient Descent with Momentum}
%\nomenclature{reLu}{Rectifying Linear Unit}
%\nomenclature{ACC}{Accuracy}
%\nomenclature{TPR}{True Psitive Rate}
%\nomenclature{FPR}{False Positive Rate}
%\nomenclature{TNR}{True Negative Rate}
%\nomenclature{FNR}{False Negative Rate}
%\nomenclature{PR}{Precision}
%\nomenclature{ROC}{Receiver Operation Characteristic}
%\nomenclature{kS/s}{Kilosamples per second}
%\nomenclature{MS/s}{Megasamples per second}
%\nomenclature{ppm}{parts-per-million}
%\nomenclature{MHz}{Megahertz}
%\nomenclature{kHz}{Kilohertz}
%\nomenclature{Hz}{Hertz}
%\nomenclature{dB}{Decibel}
%\nomenclature{NA}{Not applicable}
%\nomenclature{$\ast$}{Convolution}
%\nomenclature{$\sum{}$}{Summation Operator}
%\nomenclature{$\left|S\right|$}{Absolute value of $S$}
%\nomenclature{$\bf{C}_{XX}$}{Autocovariance function}
%\nomenclature{$\nabla$}{Gradient}
%\nomenclature{$\textbf{E}$}{Expected Value}
%\nomenclature{$\mathcal{O}()$}{Order of a Function}

%\section[Non-Intrusive Load Monitoring]{Non-Intrusive Load Monitoring}
%
%\blindtext[1]
%
%\section[Information Leakage]{Information Leakage}
%
%\blindtext[1]
%
%\section[Electromagnetic Interference Suppression]{Electromagnetic Interference Suppression}
%
%\blindtext[1]